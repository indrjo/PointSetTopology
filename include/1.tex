% !TEX program = lualatex
% !TEX spellcheck = en_GB
% !TEX root = ../topology.tex

\section{The category of Topological Spaces}

\subsection{Topologies, open/closed sets and base}

Let us start in medias res, by giving a definition which will set up our discourse.

\begin{definition}[Topological spaces]\label{definition:DefTop}
Given any set \(X\), a {\em topology}\index{topology} for \(X\) is a any collection \(\calO\) of subsets of \(X\) which satisfies:
\begin{enumerate}
\item \(\nil, X \in \calO\);
\item for every \(H \subseteq \calO\) we have \(\bigcup_{A \in H} A \in \calO\);
\item for every couple \(A, B \in \calO\) also \(A \cap B \in \calO\).
\end{enumerate}
A set \(X\) equipped with a topology \(\calO\) for it becomes a {\em topological space}\index{topological space}, its elements are called {\em points} and the elements of \(\calO\) {\em open sets}\index{open set}. 
\end{definition}

A topological space is nothing of unseen or weird: for every set \(X\) there exists its power-set \(\wp X \coloneq \set{I \mid I \subseteq X}\); consider there is \(\set{\nil, X}\) too. As you may quickly check, both those sets are topologies for \(X\). The former one is the {\em discrete topology} and the latter is the {\em indiscrete topology}.

\begin{example}
More likely, one is more acquainted with \(\reals\), who is said to have a topological structure since first courses. In fact one is told that inside \(\reals\) there are neighbourhoods of its elements, that is open intervals of the shape \((x_0-\varepsilon, x_0+\varepsilon)\), for some \(\varepsilon >0\). Clearly, we must fix all that, and make precise the claim \q{\(\reals\) has a topological structure} (we will do so later, as soon as we can). In fact the intersection of two neighbourhoods does not need to be a neighbourhood (it may be empty!), and the same holds for unions.
\end{example}

\begin{definition}[Closed sets]
For \(X\) topological space, a set \(C \subseteq X\) is said {\em closed}\index{closed set} when \(X-C\) is an open set.
\end{definition}

\begin{theorem}
For \(X\) topological space, \(\nil\) and \(X\) are closed sets. Furthermore, given any family \(\mathcal F\) of closed sets, the intersection \(\bigcap_{C \in \mathcal F} C\) is closed and the union of two closed sets is closed.
\end{theorem}

\begin{proof}
\(X = X - \nil\), so \(X\) is closed since \(\nil\) is open. Analogously, cause \(\nil = X - X\) and \(X\) is open, \(\nil\) is closed.\newline
For every \(C \in \mathcal F\), the set \(X - C\) is open. Thus, by definition, the union
\[\bigcup_{C \in \mathcal F} (X -C) = X - \bigcap_{C \in \mathcal F} C\]
is open: hence \(\bigcap_{C \in \mathcal F} C\) is closed.\newline
If \(A, B \subseteq X\) are closed, \(X-A\) and \(X-B\) are open. Again by definition,
\((X-A) \cap (X-B) = X - A \cup B\)
is open; that is \(A \cup B\) is closed.
\end{proof}

At this point, there is an important remark. A topological space is closed under the union of any collection of open sets; since we did not apply restrictions, such collections of open sets may be finite or infinite. Instead, we explicitly want a topological space to be closed under {\em binary} intersections (by induction, it is closed under finite intersection). To be honest, we don't care whether an intersection of an infinite amount of open sets is open or not.

Also, observe that saying what exactly are the closed sets in a topological space \(X\) thoroughly determines it. In fact, if you provide a family \(\calC\) which gathers precisely all its closed sets, then the set \(\calO \coloneq \set{X - A \mid A \in \calC}\) is a topology for \(X\). That is you have two manners to define topological space: you can say what is \q{open} or what is \q{closed}.

Actually, there is another way to give a complete description of a topological space: providing a base.

\begin{definition}\label{definition:DefBase}
Let \(X\) be a topological space. A {\em base}\index{base} of \(X\) is any collection \(\calB\) of its own open made as follows: for every open set \(A\) of \(X\) there exists a \(B \subseteq \calB\) such that
\[A = \bigcup_{E \in B} E\,.\] 
\end{definition}

The role of a base is clear: in a topological space an open set is exactly an union of sets of a base. The question now is: when a collection of subsets of a given set is a topology. The following lemma has an enormous importance.

\begin{theorem}[Base Lemma]\label{theorem:BaseLemma}
Let \(X\) be a set, and consider a set \(\calB \subseteq \wp X\) as follows:
\begin{enumerate}
\item for every \(x \in X\) there is a \(E \in \calB\) such that \(x \in E\);
\item for every \(A, B \in \calB\) and for every \(x \in A \cap B\) there exists \(C \in \calB\) such that \(x \in C \subseteq A \cap B\).
%\item \(X = \bigcup_{E \in \calB} E\);
%\item for every \(A, B \in \calB\) there exists a \(H \subseteq \calB\) such that \(A \cap B = \bigcup_{E \in H} E\).
\end{enumerate}
Hence the family
\[\calO \coloneq \set{E \subseteq X \mid \exists B \subseteq \calB : E = \bigcup_{I \in B} I}\]
is the unique topology for \(X\) with base \(\calB\).
\end{theorem}

\begin{proof}
We have to verify the three axioms of the definition~\ref{definition:DefTop} are satisfied.\newline
By definition of \(\calO\) himself, \(\nil\) is an element of \(\calO\).\newline
(1) is equivalent to
\[X = \bigcup_{E \in \calB} E\,,\]
so \(X \in \calO\) too.\newline
Consider now any \(\calS \subseteq \calO\) and we ask ourselves whether \(\bigcup_{E \in \calS} E \in \calO\) or not. By how \(\calO\) is made, there is an \(F(E) \subseteq \calB\), for \(E \in \calS\), such that \(E = \bigcup_{I \in F(E)} I\). Thus we have
\[\bigcup_{E \in \calS} E = \bigcup_{E \in \calS} \left(\bigcup_{I \in F(E)} I\right) = \bigcup_{I \in \bigcup_{E \in \calS} F(E)} I\]
which lies in \(\calO\), since it is regarded as an union of elements of \(\calB\).\newline
Let \(A, B \in \calO\) and study \(A \cap B\). Either of them can be rewritten as unions of elements of \(\calB\):
\[A = \bigcup_{I \in H_1} I \,, \ \ B = \bigcup_{J \in H_2} J\]
for some \(H_1, H_2 \subseteq \calB\). Then
\[A \cap B = \bigcup_{I \in H_1} \left(\bigcup_{J \in H_2} I \cap J\right)\,.\]
The condition (2) says that the intersection of two any \(A, B \in \calB\) is a suitable union of elements of \(\calB\). Take \(F(I, J) \subseteq \calB\) such that \(I \cap J = \bigcup_{E \in F(I, J)} E\), and so
\[A \cap B = \bigcup_{I \in H_1} \left( \bigcup_{J \in H_2} \left( \bigcup_{E \in F(I, J)} E \right) \right) = \bigcup_{E \in \bigcup_{I \in H_1} \bigcup_{J \in H_2} F(I, J)} E\,:\]
we can conclude that \(A \cap B \in \calO\), because \(A \cap B\) is written as union of elements of \(\calB\).
\end{proof}

\begin{example}[\(\reals^n\) with Euclidean topology]
This example explains in which sense \(\reals\), or in general \(\reals^n\), has topological structure. A \(n\)-dimensional {\em ball} or {\em sphere} with centre \(x_0 \in \reals^n\) and radius \(r \in \reals, r>0\) is the set
\[B(x_0, r) \coloneq \set{x \in \reals^n \mid \norm{x-x_0} < r}\,.\]
The family \(\set{B(x, r) \mid x \in \reals^n, r>0}\) is a base for a topology over \(\reals^n\), as you can easily show. This topology, the {\em Euclidean topology}, is the one the entire Analysis is developed upon.
\end{example}

\begin{theorem}[Equivalent bases]
Let \(X\) be a set with topologies \(\calO_1\) and \(\calO_2\). If they have respectively \(\calB_1\) and \(\calB_2\) as bases and if:
\begin{enumerate}
\item for every \(B_1 \in \calB_1\) and for every \(x \in B_1\) there is a \(B_2 \in \calB_2\) such that \(x \in B_2 \subseteq B_1\) and
\item for every \(B_2 \in \calB_2\) and for every \(x \in B_2\) there is a \(B_1 \in \calB_1\) such that \(x \in B_1 \subseteq B_2\)
\end{enumerate}
then \(\calO_1 = \calO_2\).
\end{theorem}

\begin{proof}
(1) says every element of \(\calB_1\) is an union of elements of \(\calB_2\); conversely, (2) states every element of \(\calB_2\) is an union of elements of \(\calB_1\). The proof is quite formal but simple: as exercise, you may fill in with details.
\end{proof}

\begin{exercise}
Open intervals (that is those of shape \((a, b)\) for some \(a, b \in \reals\), \(a \le b\)) forms a base for \(\reals\). Show that the family of open boxes
\[\set{\prod_{k=1}^n I_k \mid I_1, \dots{}, I_n \text{ open intervals of } \reals}\]
is a base for Euclidean topology over \(\reals^n\), as well.
\end{exercise}

Since topologies are set of sets, they can be compared in an obvious way.

\begin{definition}[Coarser and finer topologies]
Given a set \(X\) and two any topologies \(\calO_1\) and \(\calO_2\) over it, when \(\calO_1 \subseteq \calO_2\) we say \(\calO_1\) is {\em coarser}\index{coarser topology} than \(\calO_2\) or, equivalently, \(\calO_2\) is {\em finer}\index{finer topology} than \(\calO_1\). In any case we write \(\calO_2 \finer \calO_1\) or \(\calO_1 \coarser \calO_2\).
\end{definition}

Note that \(\coarser\) is the relation \(\subseteq\) restricted to topologies for the same set: thus \(\coarser\) is a partially order relation.